% Chapter 2

\chapter{Experimental Setup} % Chapter title

\label{ch:experim} % For referencing the chapter elsewhere, use \autoref{ch:examples} 

%----------------------------------------------------------------------------------------

A collection of 1322 misspellings and their corrections were obtained from Fawthrop's contribution in Birkbeck spelling error corpus \cite{mitton1985birkbeck}.
The dataset is divided into two parts: training set (comprises of 930 words) and test set (comprises of 392 words).
The hyperparameters would be tuned on the training set and then it would be tested on the test set.
Aspell dictionary \cite{atkinson2006gnu} is used as the spell-check dictionary. 
The spellings from the aspell dictionary serve as the document match for the information retrieval system.
Every query (misspelling) would have only one relevant document (correct spelling) to match.
Hence we would use MRR (Mean Reciprocal Rank) as our evaluation metric.


MeTA toolkit \cite{massung2016meta} is employed to build the search engine for the spell variants.

Following k-gram variants are employed in this experiment:
\begin{enumerate}
	\item Basic k-gram split
	\item Double-ended positional k-gram split
	\item Double-ended positional k-gram split with offsets
\end{enumerate}
The units $k = 2$, $k = 3$, and $k = 2$ with $3$ (union of $k = 2$ and $k = 3$ sets) are chosen.
Following ranking functions are employed in this experiment:
\begin{enumerate}
	\item TF-IDF, no hyperparameters \cite{tfidf}
	\item Okapi BM25 with hyperparameters $b$ and $k$ \cite{bm25}
	\item Bayesian smoothing with a Dirichlet prior with hyperparameter $\mu$ \cite{dirich}
\end{enumerate}
The experimental results are summarized in Table \ref{split}.

We used MRR (Mean Reciprocal Rank) here as our evaluation metric, since each misspelling had only one solution in our dataset.

\begin{equation*}
	MRR = \frac{1}{s} \sum_{i = 1}^s \frac{1}{rank_i}
\end{equation*}

Here $s$ is the size of the dataset and $rank_i$ is the position of the rank for the first relevant document of the $i^{th}$ query.